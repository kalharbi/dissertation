\abstract{ \OnePageChapter
	% Contribution: Methodology and supporting it by building a non trivial system.
	% 1- Deep across label (code , ui, listing, backend) and hierarchy (deep ui parents, code call graph)
	% 2- Holistic time (multiple versions) and varieties (multiple categories or scale)
	%
	Mobile app marketplaces have made it easier for developers to publish and sell apps and users to discover, install, and purchase apps. The number of apps in popular marketplaces is growing exponentially. 
	Despite the importance of marketplaces on the mobile app ecosystem, very little research has been done on indexing mobile apps to gain insights into their unique characteristics.
	Furthermore, the increase number of app updates published to the marketplace has largely gone unused by existing search engines.
	The perspective of existing marketplace search engines is generally limited to the apps' listing information and a single snapshot of the app. 
	Such view misses the much larger opportunity of mining apps with a holistic view and its utility to existing and innovative systems.

	This dissertation introduces a deep search engine called \textit{Sieveable} for mobile app marketplaces with the goal of enabling deep and holistic view analysis of mobile applications over time. 
	With Sieveable, one could quickly retrieve a sample from a large corpus of apps that meet certain criteria with respect to the listing details data, the visual design data, and the source code data in an integrated manner. This work illustrates the power of a deep and holistic view to mining apps at large-scale.
	We use Android as the application platform for the work presented here; at present, it has the largest number of apps in its official marketplace, the Google Play Store.
	
	Our goal is to inform marketplace owners, platform engineers, and third-party developers with our findings. 
	We demonstrate how Sieveable enables a set of novel data mining applications that would have been very difficult to create otherwise.
}