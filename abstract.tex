\abstract{ \OnePageChapter
	% Contribution: Methodology and supporting it by building a non trivial system.
	% 1- Deep across label (code , ui, listing, backend) and hierarchy (deep ui parents, code call graph)
	% 2- Longitudinal (multiple versions) and varieties (multiple categories or scale)
	% More apps in the market, small research studies, single version, -> Less useful results
	Data has been the foundation of science for decades.
	But it is the massive amount of data that drives our modern scientific discovery.
	Mobile applications are a major source of large amount of data that covers various aspects of our lives.
	As the number of mobile applications continue to proliferate in marketplaces, the need to fully understand various aspects of them is ever increasing.
	While mobile applications have recently received an increase in research attention, they have been rarely studied holistically over time at a massive scale.
	Furthermore, the increase number of application updates published to the marketplace has largely gone unobserved.
	The perspective of prior approaches have been generally limited to a single view and the latest snapshot of the application.
	Such view misses the much larger opportunity of mining applications with both a deep and longitudinal views and utilizing it to create innovative systems.

	This dissertation introduces a deep and longitudinal approach embodied in a scalable infrastructure called \textit{Sieveable} with the goal of enabling deep and holistic view analysis of the design and development of mobile applications over time.
	With Sieveable, one could quickly retrieve a sample from a large dataset of apps that meet certain criteria with respect to the meta-data, visual design, and source code data in an integrated manner.
	This work illustrates the power of a deep and longitudinal views to mining apps at large-scale.
	
	Our goal is to inform marketplace owners, platform engineers, and third-party developers with our findings. 
	We demonstrate how Sieveable enables different types of analyses that would have been very difficult to perform otherwise.
	We argue that considering both a holistic view and temporal view that involve design and development, result in more useful mobile app analysis as supported by our findings. 
}