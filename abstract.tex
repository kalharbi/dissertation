\abstract{ \OnePageChapter
	% Contribution: Methodology and supporting it by building a non trivial system.
	% 1- Deep across label (code , ui, listing, backend) and hierarchy (deep ui parents, code call graph)
	% 2- Longitudinal (multiple versions) and varieties (multiple categories or scale)
	%
	Mobile app marketplaces have made it easier for developers to publish and sell apps and users to discover, install, and purchase apps.
	The number of apps in popular marketplaces is growing exponentially, yet, very little research has been done on indexing mobile apps to gain insights into their unique characteristics.
	Furthermore, the increase number of app updates published to the marketplace has largely gone unobserved in prior research.
	The perspective of existing data mining approaches is generally limited to a single view and the latest snapshot of the app.
	Such view misses the much larger opportunity of mining apps with both a deep and longitudinal views and utilizing it to create innovative systems.

	This dissertation introduces a deep and longitudinal approach embodied in a search engine called \textit{Sieveable} with the goal of enabling deep and holistic view analysis of mobile applications over time. 
	With Sieveable, one could quickly retrieve a sample from a large corpus of apps that meet certain criteria with respect to the listing details,  visual design, and source code data in an integrated manner.
	This work illustrates the power of a deep and longitudinal views to mining apps at large-scale.
	We use Android as the application platform for the work presented here; at present, it has the largest number of apps in its official marketplace, the Google Play Store.
	
	Our goal is to inform marketplace owners, platform engineers, and third-party developers with our findings. 
	We demonstrate how Sieveable enables different types of analyses that would have been very difficult to perform otherwise.
}