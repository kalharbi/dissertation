\chapter{Conclusion and Future Work}
\label{ch:conclusion}

\section{Conclusion}
This dissertation introduced a deep and longitudinal approach to mining mobile applications.
I used the term ``deep'' to refer to the deep and structural indexing of apps across multiple levels, and the term ``longitudinal'' to refer to observing them over multiple points of time.
The ultimate contribution of this dissertation research is the advancement of understanding mobile apps at deep and longitudinal levels.
The work proposed here sought to gain insights into mobile apps by enabling the analysis at unprecedented depth over time.
Specifically, this dissertation aimed to answer the following two research questions:
\begin{enumerate}
	\item How can we enable deep searching and mining of mobile apps over time?
	\item What are the benefits of the deep and longitudinal approach to mining mobile apps?
\end{enumerate}
In answering the first question, a novel large datasets have been collected and led to the development and deployment of a large-scale retrieval platform called Sieveable (discussed in chapter~\ref{ch:sieveable_chapter}).
In answering the second question, I used Sieveable to present several types of analyses that would have been difficult to perform otherwise  (discussed in chapter~\ref{ch:findings_chapter}).
Some highlights of my findings include:
1) In user interface design, the release of official design libraries enabled new applications to reduce the gap with the most downloaded ones in adopting best design practices.
2) In accessibility, results showed that accessibility is a problem for many applications including the most downloaded ones.
3) In privacy, the most added permissions in each year are the ones often required by ad libraries, which raises privacy concerns.

\section{Future Work}
There are a number of promising future directions for the work presented in this dissertation.

\subsection*{Ranking Algorithms:}
The number of apps has increased dramatically in mobile marketplaces.
As a result, finding a new and well designed app is becoming an exercise in frustration.
Most if not all existing search ranking algorithms do not take into account apps' internal data when ranking search results.
Search systems miss a great opportunity in creating novel ranking algorithms based on apps' data.
Employing new ranking factors such as accessibility, design, privacy, etc. could lead to significant benefits to users.

\subsection*{Personalized Recommended Systems:}
The existing recommended systems in marketplaces do not recommend apps based on user's preferences.
It will be useful to recommend an app to a user if it matches her preferences.
Users have different visual aesthetic or privacy preferences.
One promising example of personalized recommender systems is a visual design recommender system.
By extracting visual design features of each app and learning the preferences of each user, we can recommend apps that match user's design preferences.

\section{Summary}
This dissertation introduced an approach to mining large and ever changing marketplaces by taking a deep and longitudinal perspectives.
I presented several analyses that encompassed this approach and uncovered new findings.
I believe that additional efforts in this research area could result in innovative systems that empowers mobile developers and users.
I hope this work inspires others to apply an approach with a deep and longitudinal perspectives to new problems.