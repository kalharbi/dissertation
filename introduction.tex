\chapter{Introduction}
\label{ch:intro__chapter}
Modern mobile platforms feature a distribution platform for applications called marketplaces (or app stores).
App marketplaces are online software distribution stores for developers to publish their apps for free or sell them, and for users to discover, purchase, download, and update apps.
The popularity of mobile devices and the advances in their operating systems have led to significant increase in the number of apps published in marketplaces.
For example, as of October 2016, the number of apps in the Google Play Store is over 2.4 million apps \cite{appbrain_play_apps}, which makes it the largest digital marketplace.
These apps have become a valuable data source to mine and extract insight from in both academia and industry.

In the recent years, there has been a noticeable amount of research activities on how to extract meaningful insights from apps data. 
The research in mining mobile apps has been dominated by three single views.
First, researchers have mined the meta-data or listing details data of apps such as ratings and user reviews to perform sentiment analysis and help developers make informed decisions supported by data \cite{fu_2013_KDD,chen_2014_ICSE,kong_2015_CCS}. 
Others have created tools and commercial services to assist app developers and publishers in better understanding of listing details data \cite{appfigures,applause,appannie}.
Second, researchers have mined user interface data such as styles and layouts of thousands of apps to gain insights into their design patterns \cite{shirazi_EICS_2013,Alharbi_2015_MobileHCI}.
Third, researchers have also mined the source code of apps to learn about malicious behavior and protect users' privacy-sensitive data \cite{zhou_2012_SP_dissecting,lu_2012_CCS,Arzt_2014_PLDI}.
What is missing in prior research is an approach that takes a deep, holistic view of the apps encompassing these three views.

This dissertation pursues a more deep search approach that takes a holistic view of the apps over time and can potentially accomplish what is currently not possible in a single view approach.
In user interface (UI) design analysis, UI components are often created or modified at runtime.
When only analyzing the static layout files, this observation is missed because that behavior is defined in the app source code.
This shortcoming can be solved by combining both the design and code views.
In sentiment analysis of user reviews, it is often difficult to link opinions to specific app features.
By incorporating the visual view and code view, one can potentially establish a causal relationship between a new feature (or a bug) and the onset of certain opinions.
In security analysis, a function could be determined, through program analysis, to be triggering a sensitive operation, such as sending an SMS message or taking a photo.
But it is often hard to judge if the sensitive operation is warranted from the program view alone.
By also taking a view of the design, one may examine which button may be linked to this sensitive operation and whether the button's label legitimizes such use (e.g., a button labeled ``Send'' for sending an SMS message).
Indexing apps to support multi-view data mining of apps is always challenging because it requires an infrastructure for integrating multiple heterogeneous data sources.

Mobile applications are always changing and increasingly updated at high rates.
Most prior work in this area focuses on a single snapshot approach that only tells us about the moment of the observation.
This single snapshot approach implies that all app updates and changes are gone unobserved.
Such view misses the benefits of observing the changes to the design and development of mobile apps in response to major events.
We cannot overlook the importance of the changes happening to mobile apps that are producing larger patterns and interesting insights.
To observe and extract knowledge from these changes, one needs to track historical app releases.
Collecting and analyzing mobile apps over time is challenging and requires building a scalable infrastructure to support analyzing large amount of data.
It is also hard if not impossible to collect the data and observe all the changes that occurred to mobile apps.
However, recognizing the dynamic nature of mobile apps and the value of the generated historical data necessitates collecting it even if we cannot observe all the changes.

Over a short period of time, mobile user interface design has evolved to enhance the overall user experience.
We observe changes to UI design guidelines, tools, and patterns at different points of time.
A once popular design pattern may begin to decline in popularity.
A new design pattern may be introduced with a lot of hype and promises but never gets widely adopted.
A little-known design pattern may all of a sudden gains high popularity.
These phenomena would have been missed, had we considered a single snapshot of the app that only tells us about the current moment.
In mobile security, an app may start showing normal behavior and perhaps later in a new version starts to exhibit malicious behavior.
Considering a single snapshot of the app would make security analysis less effective.
In estimating the accessibility of mobile apps, it has hard to gain useful insights that have great implications from a single snapshot approach.
One may attempt to quantify the prevalence of accessibility problems in mobile apps using the single snapshot approach.
Such attempt will only tell us about the moment of the observation and will fail to provide insights on how did you get to this observation.
When analyzing the same task but over time, we can uncover critical insights that help us understand whether these observations represent a state of improvement or deterioration.
Unfortunately, the single snapshot view of analysis only tells us about the ``current moment'' and that's not enough in today's ever-evolving mobile apps scene.
A longitudinal perspective that counts for all changes over time can tell us more than that and open myriad opportunities for further research in multiple areas.

To this end, this dissertation makes two major contributions.
First, it presents a novel approach to mining large software marketplaces such as the official Android marketplace by taking a deep and longitudinal perspectives, manifested in a scalable retrieval platform called \textit{Sieveable}.
Sieveable indexed more than four hundred thousand Android applications at an unprecedented level of depth (listing details, user interface, and code data).
Second, it demonstrates the utility of the approach taken by conducting diverse types of analyses that illustrate the benefits of the deep and longitudinal approach for mining mobile applications.

\section{Overview}
This thesis is divided into 8 chapters.
Chapter \ref{ch:conceptual_framework_chapter} discusses the main concepts and elements in a digital marketplace that drive changes.
Chapter \ref{ch:related_work_chapter} reviews prior related work in the area of mining the web, mobile applications, and software repositories.
Chapter \ref{ch:mining_design_changes_chapter} presents a pilot experiment for mining user interface design pattern changes in a small-scale dataset of Android apps.
The challenges faced during this pilot experiment led to the design of a scalable platform for mining mobile applications over time called Sieveable.
Chapter \ref{ch:sieveable_chapter} introduces Sieveable, discusses the technical requirements of designing the retrieval platform, and presents the process of indexing apps at multiple levels.
Chapter \ref{ch:queries_chapter} presents several illustrative search queries across multiple levels that demonstrate Sieveable's capabilities, and how it enables different types of deep analyses of mobile apps.
Chapter \ref{ch:findings_chapter} applies the presented approach to problems in mobile app design, accessibility, and privacy.
Finally, chapter \ref{ch:conclusion} concludes with a discussion on the main contributions of this thesis and discuses the future directions this work may take.
