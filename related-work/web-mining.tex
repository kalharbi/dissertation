\subsection{Web Mining}
Web mining is an area of research that involves the use of data mining techniques to automate the discovery and extraction of information from the World Wide Web \cite{etzioni_1996_Communication_ACM}.
The web mining research can be classified into three main categories: content mining, structure mining, and usage mining \cite{madria_1999_Springer,kosala_2000_Survey}.

Web content mining involves the use of various techniques due to the different kinds of content on the web (e.g., text, and multimedia data.).
Research in mining text-based content such as news articles often represents the unstructured text documents as bag of words or vector representation, n-gram, phrase based, or term based representation \cite{manning_2008_intro_to_IR}.
The applications of mining text content includes text classification, clustering, and finding patterns.
Mining multimedia data has been a major focus for researchers  and involve the use of various information retrieval techniques \cite{Wang_2011_SIGIR, Wu_2011_WSDM}.

Structure mining research focuses on extracting information from the underling hyperlinks graph structure of the Web itself.
Applications of web structure mining include Web pages ranking, categorization, and community discovery.
A number of algorithms have been proposed to model the graph structure of the hyperlinks.
The Hyperlink-Induced Topic Search (HITS) algorithm \cite{Kleinberg_1999_JACM}, PageRank \cite{Brin_1998_PageRank}, and Clever \cite{chakrabarti_1999_Computer} are all examples of algorithms that rates Web pages based on quality or importance factors. 

Web usage mining involves collecting and analyzing server and browser logs data that result from users interacting with Web pages.
The applications of mining usage data include modeling user profiles, collaborative filtering, recommender systems, and discovering navigation patterns.
