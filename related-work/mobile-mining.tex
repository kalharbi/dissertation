\subsection{Mining Mobile Applications}
Mining mobile apps started to be of interest to researchers (\cite{Enck_2010_OSDI,Enck_2011_USENIX,zhou_2012_SP_dissecting,Yin_2013_WSDM,Minelli_2013_CMSREuro,Pandita_2013_USENIX,gorla_2014_ICSE,chen_2014_ICSE,Liu_2014_ICSE,Lin_2014_SOUPS,rasthofer_2014_NDSS,Liu_2015_MobiSys,seneviratne_2015_SIGMobile,Yang_2015_ICSE,Baeza-Yates_2015_WSDM,Chen_2015_WSDM,Liu_2015_WSDM}, see Table~\ref{tab:table_mobile_mining_studies}).
The very first work to mine and analyze app store data was introduced by Harman et al. \cite{Harman_2012_MSR}.
They mined app meta-data (listing details data) of over 32,000 apps in the Blackberry app store to find patterns such as a correlation between app rating and download count.
Fu et al. \cite{fu_2013_KDD} conducted a sentiment analysis on the reviews of over 170,000 mobile apps to identify the common reasons why users like or hate an app. 
They collected listing details data and analyzed the reviews at multiple granularities.
They found that the lack of attractive design was the largest cause of negative reviews. 
However, only listing details data were considered and no design and code data were analyzed.
Viennot et al. \cite{viennot_2014_metrics} developed the largest scale crawler for Android apps, crawling over 1,000,000 apps and analyzing a subset of their listing details and source code to identify library usages and detect dangerous security practices.
Despite the immense scale, they stored the listing details and source code files as plain-text data and used a full-text search engine to query them.
This greatly restricted the applicability of their approach to keyword search and left structured data search unsupported (e.g., user interface layout data).
Shirazi et al. \cite{shirazi_EICS_2013} collected 400 Android apps to analyze the common design patterns of these apps.
They estimated the complexity of each app's design by counting the number of activities, layout files, and images.
They computed descriptive statistics such as the most frequent interface elements and the most common combination of widgets.
Minelli and Lanza introduced SAMOA \cite{Minelli_2013_CMSREuro}, an analytic platform for mobile apps with the goal to understand the complexity of mobile apps and third-party API usage.
The platform was among the first attempts to analyze apps at depth (source code structure and listing details) over time.
However, the platform does not utilize user interface data and used a small dataset of 20 open source apps.

In order to deal with the recent increase in the number of mobile apps, researchers have studied them for various data mining and machine learning applications.
Chen et al \cite{Chen_2015_WSDM} presented SimApp, a framework for detecting apps similarity based on listing details data using a kernel-based learning algorithm.
Lin et al. \cite{lin_2014_SIGIR} proposed a framework to improve app  recommendation by incorporating version histories with standard recommendation techniques.
Zhu et al. \cite{zhu_2013_CIKM} proposed a ranking fraud detection system based on historical app ranking data to detect fraudulent means to boost rate an app in the marketplace.
AppJoy \cite{yan_2011_MobiSys} is a personalized recommender system for Android apps.
The system computes a ``usage score'' based on the actual user's usage and uses that as an input for a collaborative filtering algorithm to make personalized recommendations.

\clearpage
\begin{longtable}{| c | p{3cm} | p{6cm} | p{2cm} | p{1.5cm} |}
	\caption{Summary of studies that involve mining of mobile application (continued on next page). \label{tab:table_mobile_mining_studies}} \\
	\hline
	\textbf{Domain} & \textbf{Study/System} & \textbf{Method/Purpose} & \textbf{Depth} & \textbf{Size} \\
	\hline
	\multirow{5}{*}{\rotatebox{90}{\kern-18em Privacy}}
	& TaintDroid by Enck et al. \cite{Enck_2010_OSDI}
	& An information flow tracking system that provides realtime monitoring of privacy sensitive data leaked by applications. 
	& Code
	& 30
	\tabularnewline
	\cline{2-5}
	& PEDAL by Liu et al. \cite{Liu_2015_MobiSys}
	& A system that enables users to control inherited permission to ad libraries, so they can grant a permission to the app to function but not the ad component within the app itself.
	& code
	& 60,000
	\tabularnewline
	\cline{2-5}
	& Seneviratne et al. \cite{seneviratne_2015_SIGMobile} 
	& Extract listing detail features for a list of user installed apps to predict user's gender.
	& Listing details
	& 4,167
	\tabularnewline
	\cline{2-5}
	& Lin et al. \cite{Lin_2014_SOUPS}
	& A static code analysis approach to analyzing requested permissions and determining which ones are needed for the app's core functionality and the ones needed for sharing sensitive information with ad libraries.
	They leveraged crowdsourcing to collect privacy preferences and identified four user privacy profiles.
	& Code and Listing details
	& 108,246
	\tabularnewline
	\cline{2-5}
	& SUSI by Rasthofer et al. \cite{rasthofer_2014_NDSS}
	& An automated machine learning approach for classifying sources of sensitive data (e.g., location) and sinks of potential channels through which data may leak to an adversary (e.g., a network connection) from Android API methods.
	& Code
	& 11,000
	\tabularnewline
	\hline
	\multirow{4}{*}{\rotatebox{90}{\kern-9em Security}} 
	& Pandita et al. \cite{Pandita_2013_USENIX} 
	& Use NLP techniques to detect whether the app description indicates the app needs to use a particular permission.
	& Description, API usage, and Manifest file.
	& 581
	\tabularnewline
	\cline{2-5}
	& Gorla et al \cite{gorla_2014_ICSE}
	& Analyze and cluster apps by their descriptions and API usages to detect outliers.
	& Description and API usage
	& 22,500
	\tabularnewline
	\cline{2-5}
	& The ded decompiler by Enck, et al. \cite{Enck_2011_USENIX}
	& A decompiler to recover source code from the binary file.
	They used it to perform security analysis and discovered suspicious behavior that links to misuse of personal information by the app and ad libraries.
	& Code
	& 1,100
	\tabularnewline
	\cline{2-5}
	& Zhou et al.\cite{zhou_2012_SP_dissecting}
	& A classification of Android malwares into 49 different families.
	& Code
	& 1,260
	\tabularnewline
	\hline
	\multirow{4}{*}{\rotatebox{90}{\kern-3em Software Engineering}} 
	& Chen et al. \cite{chen_2014_ICSE}
	& Analyze app user reviews to extract valuable information for developers.
	& User reviews
	& 4
	\tabularnewline
	\cline{2-5}
	& SAMOA \cite{Minelli_2013_CMSREuro}
	& An analytic Platform for mobile apps to analyze the structure of source code over time.
	& Listing details and source code.
	& 20
	\tabularnewline
	\cline{2-5}
	& Yang et al. \cite{Yang_2015_ICSE}
	& Control-flow analysis system for Android. It generates a callback control-flow graph that can be used for analysis applications such as software and GUI testing.
	& Code
	& 20
	\tabularnewline
	\cline{2-5}
	& PerfChecker by Liu et al. \cite{Liu_2014_ICSE}
	& A static code analyzer to detect performance bugs
	& Code
	& 29
	\tabularnewline
	\cline{2-5}
	\hline
	\multirow{4}{*}{\rotatebox{90}{\kern-14em Machine Learning}} 
	& Baeza-Yates et al. \cite{Baeza-Yates_2015_WSDM}
	& A model for predicting the next installed app the user is going to use. The goal is to improve the usage of home-screen/launcher applications.
	& Log usage data
	& 70,000
	\tabularnewline
	\cline{2-5}
	& Chen et al \cite{Chen_2015_WSDM} 
	& A framework for detecting similar apps using an online kernel algorithm.
	& Listing details
	& 21,624
	\tabularnewline
	\cline{2-5}
	& Liu et al.\cite{Liu_2015_WSDM}
	& App recommender system that incorporates both apps' functionalities and users' privacy preferences.
	& User reviews
	& 6,157
	\tabularnewline
	\cline{2-5}
	& Yin et al. \cite{Yin_2013_WSDM}
	& A recommender system for recommending new apps to replace already installed ones.
	& Description
	& 5,661
	\\
	\hline
\end{longtable}