\subsection{Mining User Interface Data}
User Interface design is often considered hard and challenging. Designers use various methods and techniques during the creative design process including: a) The use of paper prototypes \cite{newman_2000_DIS, klemmer_2001_UIST}. 
b) The use of storyboarding and wireframing \cite{newman_2000_DIS, wahid_2011_CHI}. 
c) The use of reusable design solutions (e.g., templates, design patterns) \cite{jacobs_2003_TOG, gibson_2005_WWW, ritchie_2011_CHI}.
d) The use of interactive sketching tools \cite{landay_1995_CHI, lin_2002_CHI, Newman_2003_HCI, sezgin_2006_GRAPH}.
e) The use of inspirational curated examples \cite{purcell_1992_KBS, herring_2009_CHI, lee_2010_CHI, ritchie_2011_CHI, miller_2014_ASME}.
Several lines of research were developed around these methods and largely used small controlled studies to evaluate their effectiveness.

Prior research have studied the use of examples to draw inspirations and aid designers in the design process.
The benefits of design examples are often cited as: 
a) Exploring alternative designs, 
b) evaluate a design against other designs, 
c) improving the communication between designers and clients, 
d) identifying the novelty of the design, 
e) An effective method for validating the design \cite{purcell_1992_KBS, herring_2009_CHI, lee_2010_CHI, ritchie_2011_CHI, chang_2012_CHI, miller_2014_ASME}.
Herring et al. \cite{herring_2009_CHI} studied how designers use examples and the difficulties they faced when searching, sharing, and using design examples.
Miller et al. \cite{miller_2014_ASME} studied how professional designers find design examples.
They found that designers had several difficulties to find what they are looking for and turning that into a search query.
This forces designers to use search terms that are not directly related to what they are looking for.
For instance, a search query for the term ``large navbar'' usually returns results to question-and-answer websites or tutorials that describe the implementation details of ``navbars''.
These findings suggested the need for better tools and retrieval systems to support designers in retrieving example.


Attempting to address this pressing need, data-driven approaches have been applied to identifying design examples.
HCI researchers have realized that in order to increase the usefulness of design examples to designers, a new generation of design based search engines needs to be developed.
This manifested from the limitations of existing search engines that are designed to deal with text rather than visual structure.

Several tools have been developed to aid designers in finding and using a curation of design examples.
The Adaptive Ideas tool \cite{lee_2010_CHI} is a browser extension that enables designers to view design examples of a manually curated corpus of 250 web design examples.
It allows designers to borrow design elements from multiple design examples while designing their websites.
D.tour \cite{ritchie_2011_CHI} is a search tool for finding web design examples.
The tool consists of a curated database of 300 web pages.
Users can search for similar design examples or use textual design terms.
WebCrystal \cite{chang_2012_CHI} is a tool that adds visualization and text to explain how design examples are constructed allowing novice web designers to use them in their own web pages.
Bricolage \cite{kumar_2011_CHI} is an algorithm for retargeting content between web pages.
It creates a mapping between the visual elements of web pages allowing designers to transfer the style from one page to another page.
Bricolage trains a model on a corpus of design mapping of 50 web pages collected from crowdsourced workers to automatically transfer the design between web pages.
Webzeitgeist \cite{kumar_2013_CHI} is a large-scale platform for mining design data comprising a dataset of over 100,000 web pages.
It provides a custom JSON-like design query language (DQL) that can be used to find design examples by the visual appearance and the DOM structure of web pages.


The HCI literature has also examples on the use of crowds to complement the data-driven approaches in finding more relevant design examples \cite{kumar_2011_CHI, spirin_2014_WWW}, understand aesthetic preferences \cite{reinecke_2013_CHI}, and design demographics \cite{reinecke_2014_CHI}. 
Reinecke et al. \cite{reinecke_2013_CHI} collected 450 web pages and obtained subjective ratings from 548 volunteers about the visual complexity and colorfulness of these pages.
They were able to develop computational models that accurately measure the perceived visual complexity and colorfulness of website screenshots.
Extending this work, Reinecke and Gajos \cite{reinecke_2014_CHI} collected 2.4 million ratings from 40,000 diverse participants for 430 websites to identify the demographical factors that influence visual preferences. 
They developed a computational model that predicts users perception of visual aesthetics for specific demographic group.

The web search and data mining literature had a different goal when dealing with design data, removing design data to improve the accuracy of information retrieval systems.
Design data is often considered unclean data that pollutes the content; thus, research has been conducted to detect and remove them to improve page ranking, indexing, and data mining algorithms \cite{bar_2002_WWW, gibson_2005_WWW, chakrabarti_2007_WWW}. 
Researchers have also developed algorithms to extract web page templates and understand their evolution.
These algorithms primarily target web browsers (e.g., template caching), search engines (e.g., improve indexing), and data mining applications (e.g., analytic tools).
For instance, Gibson et al. \cite{gibson_2005_WWW} extracted templates from web pages and measured their prevalence.
They found that templates are counted for 40-50\% of the size of web pages.
They further studied changes to the amount of templates for 183 websites over a period of 8 years and showed that it is growing by 6-8\%.