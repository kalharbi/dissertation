\section{Design Aesthetics}

Early on, researchers realized the need to formally develop design guidelines to improve the overall appealing of graphical user interfaces (GUIs) \cite{Gould_1983_CHI, hewett_1986_CHI}.
Platform owners and organizations also developed their own design guidelines with recommendations to solve problems like usability and inconsistency in GUIs and promote a good visual design appealing \cite{Web_Design_W3C,Design_for_Android, Design_for_iOS, Design_for_Windows}.
While design guidelines are sometimes useful to designers, it is now acknowledged that design guidelines are vague, conflicting, and difficult to apply \cite{borges_1996_CHI_Guidelines,de_1990_INTERACT_Guidelines,lowgren_1992_CHI_knowledge,ivory_2001_CHI_empirically}.
In addition, there is no common agreement on what constitutes valid design guidelines \cite{ratner_1996_CHIComp_characterization}.
Research efforts have moved to studying how existing web design examples influence users in many ways: visual aesthetics preferences \cite{reinecke_2011_TOCHI,reinecke_2013_CHI,reinecke_2014_CHI}, perceived trustworthiness \cite{Basso_EC_2001,lindgaard_2011_TOCHI_trustworthiness}, usability \cite{Gould_1983_CHI,tractinsky_1997_CHI_aesthetics,krug_2005_book,van_2008_IC_modelling}, quality \cite{hartmann_2008_TOCHI}, and satisfaction \cite{teo_2003_HCS_empirical}.

Researchers have conducted a series of controlled laboratory experiments with users and professional experts to understand the qualitative factors that influence users when judging the appearance of a website \cite{teo_2003_HCS_empirical, lavie_2004_HCS_assessing, hartmann_2008_TOCHI}. 
Lindgaard et al. \cite{lindgaard_2006_BIT_attention} conducted three lab studies to find how quickly users assess the attractiveness of website design and found that participants were able to judge the design within 50 milliseconds.
Hartmann et al \cite{Hartmann_CHI_2008} conducted a small controlled lab study to evaluate the attractiveness of websites by applying the adaptive decision making theory.
They collected subjective ratings from 43 participants for only 3 websites and found that participants background influenced their rating of website design quality.
Researchers have also applied different techniques to measure web design aesthetics. 
For instance, Ivory et al. \cite{ivory_2001_CHI_empirically} used a quantitative method to compute 11 web page attributes metrics (e.g., number of fonts, images, and words) for 1,898 web pages from 163 websites.
The web pages were manually obtained from a list of websites awarded an award for excellence in web design.
They achieved an accuracy of predicting 65\% of the award judgment ratings.
Ivory et al. \cite{ivory_statistical_2002} expanded upon this work by adding more 146 attribute metrics, increasing the number of samples to 5,300 web pages, achieving a higher accuracy of 94\%, and creating a profile
of good and bad website design.

Advances in computer vision algorithms enabled researchers to
apply pixel-based methods to analyze the design of websites and develop computational models to predict users judgments.
Zheng et al. \cite{zheng_2009_CHI_correlating} computed low-level image statistics to predict users' judgments for 30 web pages.
They computed layout structures and evaluated them with human ratings collected from a study with 22 participants.
Reinecke et al. \cite{reinecke_2013_CHI} collected 450 web pages and obtained subjective ratings from 548 volunteers for the visual complexity and colorfulness of these web pages.
They were able to develop computational models that accurately measure the perceived visual complexity and colorfulness of website screenshots.

Recently, Miniukovich and De Angeli \cite{Miniukovich_CHI_2015} developed a tool for GUI aesthetics evaluation based on eight metrics of GUI aesthetics.
The tool was evaluated in two lab studies for 62 web pages and 53 mobile applications. 
