\section{Android Applications}
Android applications are often written using the Android software kit (SDK) in the Java programming language.
In addition, parts of the application code can be written in native languages such as C and C++ using the Native Development Kit (NDK).
Android provides a rich framework with various APIs for third party developers and a set of tools to build and compile the application source code along with its resources into an Android package file (APK).
The APK file is an archive file in ZIP format with a .apk file extension.
An application developer generates a certificate to digitally sign the application before releasing it on the marketplace for Android devices.
When publishing the app to the Google Play store, the official marketplace, the developer provides information for the store listing to describe and promote the application.
Users can search for applications in the marketplace and install the APK file to their devices.
Applications in the marketplace are uniquely identified using a package name field.
Android uses the Java package naming conventions to uniquely identify applications by their package names.
In general, developers use a package name that begins with their reversed Internet domain name (e.g., com.airbnb.android).
Developers can update the APK file with a newer version and change the store listing details page at any time.
The Google Play store only offers the most recent version of the app and does not keep previously submitted versions.
However, some unofficial or third-party marketplaces maintain an archive for all versions of particular apps.
In addition to package names, Android requires applications to define version information, so it can be used when releasing or upgrading the APK file. Android uses two version values. 
a) version name, which is a string value that represents the application release version and is visible to users (e.g., 1.2.0), and b) version code, which is an integer value that is not visible to the users but used by marketplaces including the Google Play store to check for version updates.

\subsection{User Interface Layout}
The user interface (UI) of Android applications is built using two basic UI elements, View and ViewGroup elements.
Views represent a single UI component (e.g., input elements such as text view, button, etc.), while ViewGroups are the invisible containers that group View elements (e.g., LinearLayout, RelativeLayout, etc.).
UI elements are usually defined in static XML layout files. The app content (e.g., text, images, animations, etc.) is usually stored in XML files within special directories and embedded using special XML tags and attributes.
The Android framework parses the layout XML files into a Document Object Model (DOM) tree, performs pre-processing on the tree at build time, and inflates the screen with the visual rendering of the layout.
The XML layout files provide a complete specification for the Android framework engine to render the user interface.
The UI of Android applications can be constructed entirely at runtime; however, for most applications the UI is implemented in static XML layout files, and Java code is used to add content or enhance interactive UI elements.
For example, to implement a navigation drawer, the developer would need to define the drawer layout, initialize the drawer list, and handle navigation click events (see listing~\ref{lst:ui_example}).

\begin{listing}[!htb]
	\caption{A snippet of an Android drawer menu defined in an XML file and initialized in Java code.}
	\begin{minted}{xml}
$ main_layout.xml
<android.support.v4.widget.DrawerLayout
    xmlns:android="http://schemas.android.com/apk/res/android"
    android:id="@+id/drawer_layout"
    android:layout_width="match_parent"
    android:layout_height="match_parent">
    <FrameLayout
        android:id="@+id/content_frame"
        android:layout_width="match_parent"
        android:layout_height="match_parent" />
    <ListView android:id="@+id/main_drawer"
        android:layout_width="match_parent"
        android:layout_height="match_parent"
        android:background="#111"/>
</android.support.v4.widget.DrawerLayout>
\end{minted}
\begin{minted}{java}
$ MainActivity.java
public class MainActivity extends Activity {
   private String[] mMenuItems;
   private DrawerLayout mDrawerLayout;
   private ListView mDrawerList;
   @Override
   public void onCreate(Bundle savedInstanceState) {
      super.onCreate(savedInstanceState);
      setContentView(R.layout.activity_main);
      mMenuItems = getResources().getStringArray(R.array.menu_items);    
      mDrawerLayout = (DrawerLayout) findViewById(R.id.main_layout);
      mDrawerList = (ListView) findViewById(R.id.main_drawer);
				
      mDrawerList.setAdapter(new ArrayAdapter<String>(this,
             R.layout.drawer_list_item, mMenuItems));
				
      mDrawerList.setOnItemClickListener(new DrawerItemClickListener());
   }
}
\end{minted}
	\label{lst:ui_example}
\end{listing}

\subsection{Reverse Engineering Android Applications}
According to the taxonomy of Chikofsky and Cross \cite{chikofsky_1990_reverse}, reverse engineering is defined as ``the process of analyzing a subject system to identify the system's components and their interrelationships and create representation of the system in another form or at a higher level of abstraction.''
The analysis presented in this thesis is performed on third-party, closed source, binary Android applications.
In order to expose the app’s structure, we make use of a reverse engineering tool called \textit{apktool} \cite{apktool} to unpack APK files and decode their resources to their nearly original formats.
Running \textit{apktool} on an APK file results in a directory tree that makes up the app.
The tool disassembles the byte code into \textit{smali} files, which uses an assembler-like syntax based on Jasmin \cite{jasmin_assembler}, a largely used assembly format for the Java Virtual Machine (JVM).

