\subsection{Mining Software Repositories}
Software repositories such as version control, bug tracking, and team communication systems have received a lot of attention in recent years.
Mining software repositories (MSR) is an emerging research area concerned with finding patterns in large software repositories.
It has seen a wide range of topics including detecting code redundancy \cite{Kawrykow_2009_ASE,Carzaniga_2015_ICSE}, finding relevant code examples \cite{Holmes_2005_ICSE,Sahavechaphan_2006_OOPSLA,Mandelin_2005_PLDI,Stylos_2006_VLHCC}, bug-introducing changes \cite{Sunghun_ASE_2006}, bug fixes \cite{Kim_2006_FSE,Osman_CSMR-WCRE_2014}, mining software changes \cite{Zimmermann_2005_SE,Ray_2015_MSR}, identifying hard to change code (code decay \cite{Eick_2001_SETransaction}), and finding common idioms \cite{fast_2014_CHI}.

Maintaining large collections of open source software repositories and indexing them has recently become an active research area (e.g., \cite{bajracharya_2006_OOPSLA,Caneill_2014_ESEM,Lee_2010_FSE,Stolee_2014_TOSEM}).
For instance, Boa \cite{Dyer_2013_ICSE} is a domain specific language for mining software repositories on a large-scale infrastructure with the goal of reducing the efforts of writing analyses tasks and reproducing practical mining experiments.
