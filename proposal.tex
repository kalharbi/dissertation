\chapter{Proposed Work}

\section{Introduction}
\label{ch:propsal}
The expected main contribution of this dissertation research is the advancement of understanding mobile apps at deep and holistic levels.
The work proposed here aims at gaining more insights into mobile apps by enabling the analysis at unprecedented depth and over time.
Specifically, this dissertation aims to answer the following research questions:

\begin{enumerate}
	\item How can we enable deep and holistic searching and filtering of mobile apps?
	\item What are the benefits of mining mobile apps at depth and holistic levels?
\end{enumerate}
% For q2:faster, reliable,effiecnt

In answering the first question, a novel large-scale datasets have been collected and led to the development and deployment of systems (discussed in chapter \ref{ch:mining_design_changes_chapter} and \ref{ch:sieveable_chapter}).
In this chapter, I expand on answering the second question by discussing my plan for evaluating the main contribution of this dissertation and the path that will be taken for future work, which will further expand our understanding of mobile apps.
In the next section, I discuss two main phases that I plan to work on to highlight the significance of the main contribution presented in this thesis.

%TODO: Talk about the implications of answering these two research questions on search engines and data mining applications.

\section{Phase 1: Evaluation}
I have empirically demonstrated that Sieveable can enable a deep and holistic search for mobile apps for various applications (see chapter \ref{ch:findings_chapter}).
To further support the significant of a deep and holistic views of searching mobile apps at large-scale, a comprehensive and careful evaluation is proposed here.
Specifically, the evaluation should answer the following two questions:
\begin{enumerate}
	\item How do we measure the effectiveness of the deep search results?
	\item What could a deep and holistic view of apps search accomplish what is currently not possible in a single view?
\end{enumerate}

To answer the first question, I plan to use standard basic measures for information retrieval effectiveness: precision and recall \cite{manning_2008_intro_to_IR}.
Precision refers to the fraction of retrieved results that are relevant, while recall refers to the fraction of relevant results that are retrieved.

To answer the second question, I propose conducting a comparison between the proposed approach with the following previously applied single-view approaches:
\begin{itemize}
	\item Listing Details Analysis:
	\item User Interface Analysis:
	\item Code Analysis:
\end{itemize}
The comparison will show that current single-view analysis methods are shallow and narrow, which result in missing several important observations.

\section{Phase 2: Study}
The second phase proposes a systematic study on incorporating the deep and holistic view to perform app search ranking.
\section{The Significance of the Proposed Work}

\section{Timeline}


